\documentclass{article}
\usepackage{amsmath} % Pacote para usar fórmulas matemáticas
\usepackage{graphicx} % Para incluir gráficos

\title{Resumo Geral de Estatística Descritiva}
\author{Autor}
\date{}

\begin{document}

\maketitle

\section*{Introdução}
A estatística descritiva envolve métodos para organizar, resumir e apresentar dados de maneira informativa. Os dados podem ser apresentados por meio de gráficos, tabelas e medidas numéricas.

\section*{Medidas de Tendência Central}

\subsection*{1. Média}
A média é o valor médio de um conjunto de dados. Para calcular a média, somam-se todos os valores e divide-se pela quantidade de valores.

\begin{equation}
\text{Média} = \frac{\sum x_i}{n}
\end{equation}
onde \(x_i\) representa cada valor e \(n\) é o número de observações.

\textbf{Exemplo}: Se as alturas de 5 pessoas são 160, 165, 170, 175 e 180 cm, a média é:
\[
\frac{160 + 165 + 170 + 175 + 180}{5} = 170 \text{ cm}
\]

\subsection*{2. Mediana}
A mediana é o valor central de um conjunto de dados ordenados. Se houver um número ímpar de observações, a mediana é o valor do meio. Se houver um número par, é a média dos dois valores centrais.

\textbf{Exemplo}: Para os dados 150, 160, 170, 180, 190, a mediana é 170. Para os dados 150, 160, 170, 180, a mediana é:
\[
\frac{160 + 170}{2} = 165
\]

\subsection*{3. Moda}
A moda é o valor que aparece com maior frequência em um conjunto de dados. Um conjunto de dados pode não ter moda, ter uma única moda (unimodal) ou várias modas (bimodal ou multimodal).

\textbf{Exemplo}: No conjunto de dados 1, 2, 2, 3, 4, 4, 4, a moda é 4, pois aparece mais vezes.

\section*{Medidas de Dispersão}

\subsection*{1. Variância}
A variância mede a dispersão dos dados em relação à média. Para a população, é dada por:
\begin{equation}
\sigma^2 = \frac{\sum (x_i - \mu)^2}{N}
\end{equation}
onde \(x_i\) são os valores, \(\mu\) é a média populacional e \(N\) é o número de elementos da população.

Para uma amostra, a fórmula é:
\begin{equation}
s^2 = \frac{\sum (x_i - \bar{x})^2}{n-1}
\end{equation}
onde \(\bar{x}\) é a média amostral e \(n\) é o número de elementos da amostra.

\subsection*{2. Desvio Padrão}
O desvio padrão é a raiz quadrada da variância e fornece uma medida da dispersão no mesmo nível dos dados.

\begin{equation}
\sigma = \sqrt{\frac{\sum (x_i - \mu)^2}{N}}
\end{equation}

\textbf{Exemplo}: Se os valores de uma amostra são 10, 12, 23, 23, 16, a variância e o desvio padrão podem ser calculados para verificar o quão espalhados estão em relação à média.

\subsection*{3. Intervalo (Range)}
O intervalo é a diferença entre o maior e o menor valor de um conjunto de dados:
\begin{equation}
\text{Intervalo} = x_{\text{máx}} - x_{\text{mín}}
\end{equation}

\section*{Medidas de Posição}

\subsection*{1. Percentis}
Os percentis dividem os dados em 100 partes iguais. Por exemplo, o 50º percentil corresponde à mediana.

\subsection*{2. Quartis}
Os quartis dividem os dados em quatro partes iguais. O primeiro quartil (Q1) corresponde ao 25º percentil, o segundo quartil (Q2) é a mediana (50º percentil), e o terceiro quartil (Q3) é o 75º percentil.

\section*{Conclusão}
Estatísticas descritivas são fundamentais para resumir e interpretar conjuntos de dados. Medidas de tendência central, dispersão e posição fornecem diferentes formas de analisar a distribuição dos dados.

\end{document}
